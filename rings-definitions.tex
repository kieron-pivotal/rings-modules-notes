\section{Rings}

\subsection{Definitions}

  \begin{Def}
    A \emph{binary operation} on a set $S$ is a map $\phi:S \times S \to S$.
    Similarly, a \emph{unary operation} is a map $\psi:S \to S$.
  \end{Def}

  \begin{Def}
    A \emph{semigroup} is a set $S$ with a binary operation $*$ satisfying
    the associative law:
    \[
      a * (b * c) = (a * b) * c \text{\qquad for all } a, b,c \in S
    \]

    By the definition of the binary operation, the semigroup is closed with
    respect to the $*$ operator.

  \end{Def}

  \begin{Def} A \emph{group} is a non-empty set $G$ with a binary
    operation $*$, a unary operation $x \to \bar{x}$,
    and an element $e$, such that:
    \begin{enumerate}[label=(\roman*)]
      \item $G$ is a semigroup with respect to $*$,
      \item $a * e = e * a = a$ for all $a \in G$, and
      \item $a * \bar{a} = \bar{a} * a = e$ for all $a \in G$.
    \end{enumerate}
    $e$ is the identity element of $G$, and $\bar{a}$ is the inverse of $a$.
    If the $*$ operation is commutative, the group is known as an
    \emph{Abelian} group.

  \end{Def}

  \begin{Lemma}
    Both the identity element of a group, and the inverse of a particular
    element are uniquely determined.
  \end{Lemma}

  \begin{proof}
    Suppose $e$ is an identity element of a group $G$. Let $a$ be an identity.
    Then we have $a = ae$, since $e$ is an identity, and because $a$ is an
    identity, $ae = e$, and so $a = e$, uniquely determined the identity.

    For the second part, let $a^{-1}$ be the inverse of the element $a \in G$,
    and suppose $b$ is also the inverse of a. Then we have $a * b = e = a *
    a^{-1}$, and $ a^{-1} * a * b = a^{-1} * a * a^{-1} $, so $ (a^{-1} * a) *
    b = (a^{-1} * a) * a^{-1} $ and hence $b = a^{-1}$ and the inverse is
    uniquely determined.
  \end{proof}

  \begin{Def}
    A \emph{ring} is a set $R$ equipped with two binary operations connected by
    distributive laws. $R$ is an Abelian group with respect to one binary
    operation (by convention known as addition and denoted by $+$), and is a
    semigroup with respect to the other binary operation (called multiplication
    and denoted by juxtaposition).

    The full set of ring axioms are as follows for a ring $R$:
    \begin{enumerate}[label=(\roman*)]
        \item
          $ a + b = b + a $ for all $ a, b \in R$,
        \item
          $ a + (b + c) = (a + b) + c $ for all $ a, b, c \in R$,
        \item
          There is an element $0 \in R$, such that
          $a + 0 = 0 + a = a$ for all $a \in R$,
        \item
          For each $a \in R$, there exists an element $-a$ such that
          $a + -a = -a + a = 0$,

        \item
          $ a(bc) = (ab)c $ for all $ a, b, c \in R$,

        \item
          $ a(b+c) = ab + ac $ for all $ a, b, c \in R$,
        \item
          $ (a+b)c = ac + bc $ for all $ a, b, c \in R$,

    \end{enumerate}
  \end{Def}

  \begin{Lemma}
    Some consequences of the ring axioms. If $r, s \in R$, then:
    \begin{align*}
      r0 &= 0r = 0,\\
      (-r)s &= r(-s) = -(rs),\\
      (-r)(-s) &= rs.
    \end{align*}
  \end{Lemma}

  \begin{proof}
    Since $0$ is the additive identity, we have $0 + 0 = 0$,
    hence $r(0+0) = r0$, and by the left associative law
    $ r0 + r0 = r0 $, and $r0 + r0 = r0 + 0$. Then by the
    cancellation law, $r0 = 0$. Similarly, $0r = 0$.

    Then, given $-r$ is the additive inverse of $r$, $r + (-r) = 0$ and so
    $(r + (-r))s = 0s$, and by the right distributive law and the previous
    identity, $rs + (-r)s = 0$. Now $rs + -(rs) = 0$, so
    $rs + (-r)s = rs + -(rs)$ and by the cancellation law, $(-r)s = -(rs)$.
    Similarly, $r(-s) = -(rs)$.

    By applying (2) repeatedly, we have $(-r)(-s) = -(r(-s)) = --(rs)$.
    Now, for any $t \in R$, $-t$ is the unique solution to $t + x = 0$.
    So $(-t) + t = 0$ tells us $--t = t$, and so $(-r)(-s) = rs$.
  \end{proof}

  \subsection{Special Classes of Rings}

  \begin{Def}
    A \emph{Commutative Ring} is a ring in which multiplication is commutative,
    i.e. $ab = ba$ for all $a, b \in R$.
  \end{Def}

  \begin{Def}
    A \emph{Ring with Multiplicative Identity} or \emph{Ring with 1} is
    a ring containing the element $1$, where $1a = a1 = a$ for all
    $a \in R$.
  \end{Def}

  Note that the ring $\lbrace 0 \rbrace$ is a ring with 1, and `1' in this case is $0$.

  \begin{Lemma}
    The multiplicative identity in a ring with 1 is uniquely determined
  \end{Lemma}
  \begin{proof}
    Let $e$ be the multiplicative identity. Suppose there is an $x \in R$ such
    that $rx = xr = r$ for all $r \in R$, then $x = ex = e$
  \end{proof}

  \begin{Def}
    A \emph{zero-divisor} of a ring $R$ is an element $r \neq 0 \in R$ such
    that $rs = 0$ for some $s \neq 0 \in R$.
  \end{Def}

  \begin{Def}
    An \emph{Integral Domain} is a commutative ring with $1 \neq 0$ and no zero-divisors.
  \end{Def}

  \begin{Lemma}
    Let $a$ be a non-zero element in an integral domain, and $x, y \in R$. Then
    $$ax = ay \implies x = y$$
  \end{Lemma}

  \begin{proof}
    If $ax = ay$ for $a \neq 0 \in R$ where $R$ is an integral domain, then
    $a(x-y) = 0$. Since $R$ has no zero-divisors and $a \neq 0$, $x-y = 0$,
    and so $x = y$.
  \end{proof}

  \begin{Def}
    A \emph{field} is a commutative ring in which the set of non-zero elements
    forms a group under multiplication.
  \end{Def}

  \begin{Lemma}
    A field is an integral domain.
  \end{Lemma}

  \begin{proof}
    We need to show a field $F$ is a ring with $1$, and that there are no zero-divisors.

    Now since the non-zero elements of $F$ form a group under multiplication,
    there exists an element $1 \in F$ such that $1f = f1 = f$ for all $f \neq 0 \in F$.
    Since $1 \cdot 0 = 0$ as shown above, $1$ is indeed the multiplicative identity.

    Now suppose $ab = 0$ for some $a, b \in F, a \neq 0$. Since non-zero elements of $F$
    form a group under multiplication, there exists an element $a^{-1}$ such that
    $a^{-1}a = 1$. So $a^{-1}ab = a^{-1}0$, hence $b = 0$. Thus $F$ has no zero-divisors.
  \end{proof}

  \subsection{Examples of rings}

  \subsubsection{$n\mathbb{Z}$}
  If $n \in \mathbb{Z}$, the set $n\mathbb{Z} = \lbrace a \in \mathbb{Z} : n | a \rbrace$
  is a commutative ring.

  \subsubsection{$\mathbb{Z}_{n}$}
  Given $n \in \mathbb{Z}$, define the equivalence relation $\sim$ by
  \[
    a \sim b \iff a - b = kn
  \]
  for some $k \in \mathbb{Z}$.

  Denote the equivalence class containing $a$ as $[a]$.
  Then $[0], [1], ..., [n-1]$ is the complete set of equivalence classes with respect
  to $\sim$. These are called the \emph{congruence classes modulo $n$}, and the set of
  them is denoted $\mathbb{Z}_{n}$.

  When we define the operators as follows:
  \begin{align*}
    [a] + [b] &= [a + b]\\
    [a][b] &= [ab]
  \end{align*}
  we get a commutative ring with $1$. If $n$ is composite, we have zero-divisors, so
  the ring is not an integral domain, but if $n$ is prime, $\mathbb{Z}_{n}$ is a field.

  \subsubsection{Abelian group with null multiplication}
  Let $G$ be an Abelian group, and define multiplication as $ab = 0$ for all $a, b \in G$,
  then it forms a commutative ring.

  \subsubsection{$\mathbb{C}$}
  The complex numbers form a field under the normal operations. The subrings
  $\mathbb{R}$ and $\mathbb{Q}$ also form fields, and the subring $\mathbb{Z}$
  is an integral domain.

  \subsubsection{Gaussian integers}
  ${a + bi : a, b \in \mathbb{Z}}$ forms an integral domain under the normal operations.

  \subsubsection{Power set}
  Given a set $S$, its power set is defined as the set of all subsets of $S$,
  including $S$ and the empty set. Define the operations as
  \begin{align*}
    A + B &= (A \cup B) \setminus (A \cap B)\\
    AB &= A \cap B
  \end{align*}
  and we get a commutative ring.

  \subsubsection{The set of $n \times n$ matrices}
  Let $\mathbf{M}_{n}(\mathbf{k})$ denote the set of $n \times n$ matrices over
  a field $\mathbf{k}$. The operations are defined the normal way as $(A +
  B)_{ij} = A_{ij} + B_{ij}$, and $(AB)_{ij} = \sum_{k=1}^{n} A_{ik} B_{kj}$,
  where $A_{ij}$ is the $i$th row and $j$th column of the matrix $A$. Then
  $\mathbf{M}_{n}(\mathbf{k})$ forms a ring with $1$. Note, it is not
  commutative when $n > 1$.

  \subsubsection{The set of all maps $\phi : S \to \mathbb{R}$}


  %\subsubsection

