\documentclass{article}
\usepackage{amssymb,amsmath,amsthm}

\theoremstyle{definition}
\newtheorem{definition}{Definition}[section]
\newtheorem{theorem}{Theorem}[section]
\newtheorem{lemma}[theorem]{Lemma}

\begin{document}

\begin{section}{Rings}

  \begin{definition}
    A \emph{binary operation} on a set $S$ is a map $\mu:S \times S \to S$.
    Similarly, a \emph{unary operation} is a map $\mu:S \to S$.
  \end{definition}

  \begin{definition}
    A \emph{semigroup} is a set $S$ with a binary operation $*$ satisfying
    the associative law:
    \[
      a * (b * c) = (a * b) * c \text{\qquad for all } a, b,c \in S
    \]

    By the definition of the binary operation, the semigroup is closed with
    respect to the $*$ operator.

  \end{definition}

  \begin{definition} A \emph{group} is a non-empty set $G$ with a binary
    operation $*$, a unary operation $x \to \bar{x}$ ,
    and an element $e$, such that:

    \begin{enumerate}
      \item $G$ is a semigroup with respect to $*$,
      \item $a * e = e * a = a$ for all $a \in G$, and
      \item $a * \bar{a} = \bar{a} * a = e$ for all $a \in G$.
    \end{enumerate}

    $e$ is the identity element of $G$, and $\bar{a}$ is the inverse of $a$.

    If the $*$ operation is commutative, the group is known as an
    \emph{abelian} group.

  \end{definition}

  \begin{lemma}
    Both the identity element of a group, and the inverse of a particular
    element are uniquely determined.
  \end{lemma}

  \begin{proof}
    Suppose $e$ is an identity element of a group $G$. Let $a$ be an identity.
    Then we have $a = ae$, since $e$ is an identity, and because $a$ is an
    identity, $ae = e$, and so $a = e$, uniquely determined the identity.

    For the second part, let $a^{-1}$ be the inverse of the element $a \in G$,
    and suppose $b$ is also the inverse of a. Then we have $a * b = e = a *
    a^{-1}$, and $ a^{-1} * a * b = a^{-1} * a * a^{-1} $, so $ (a^{-1} * a) *
    b = (a^{-1} * a) * a^{-1} $ and hence $b = a^{-1}$ and the inverse is
    uniquely determined.
  \end{proof}

  \begin{definition}
    A \emph{ring} is a set $R$ equipped with two binary operations connected by
    distributive laws. $R$ is an abelian group with respect to one binary
    operation (by convention known as addition and denoted by $+$), and is a
    semigroup with respect to the other binary operation (called multiplication
    and denoted by juxtaposition).

    The full set of ring axioms are as follows for a ring $R$:
    \begin{enumerate}
        \item
          $ a + b = b + a $ for all $ a, b \in R$,
        \item
          $ a + (b + c) = (a + b) + c $ for all $ a, b, c \in R$,
        \item
          There is an element $0 \in R$, such that
          $a + 0 = 0 + a = a$ for all $a \in R$,
        \item
          For each $a \in R$, there exists an element $-a$ such that
          $a + -a = -a + a = 0$,

        \item
          $ a(bc) = (ab)c $ for all $ a, b, c \in R$,

        \item
          $ a(b+c) = ab + ac $ for all $ a, b, c \in R$,
        \item
          $ (a+b)c = ac + bc $ for all $ a, b, c \in R$,

    \end{enumerate}
  \end{definition}

\end{section}
\end{document}
