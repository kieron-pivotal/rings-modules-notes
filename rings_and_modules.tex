\documentclass[a4paper]{amsart}

\usepackage{amsfonts}
\usepackage{amsthm}
\usepackage{amsmath}
\usepackage{amscd}
\usepackage[latin2]{inputenc}
\usepackage{t1enc}
\usepackage[mathscr]{eucal}
\usepackage{indentfirst}
\usepackage{graphicx}
\usepackage{graphics}
\usepackage{pict2e}
\usepackage{epic}
\numberwithin{equation}{section}
\usepackage[margin=2.9cm]{geometry}

\def\numset#1{{\\mathbb #1}}

\theoremstyle{plain}
\newtheorem{Th}{Theorem}[section]
\newtheorem{Lemma}[Th]{Lemma}
\newtheorem{Cor}[Th]{Corollary}
\newtheorem{Prop}[Th]{Proposition}

\theoremstyle{definition}
\newtheorem{Def}[Th]{Definition}
\newtheorem{Conj}[Th]{Conjecture}
\newtheorem{Rem}[Th]{Remark}
\newtheorem{?}[Th]{Problem}
\newtheorem{Ex}[Th]{Example}

\begin{document}

\title{Hartley and Hawkes - Rings, Modules and Linear Algebra - Notes}
\author{Kieron Browne}

\maketitle
\tableofcontents

\section{Rings}

\subsection{Definitions}

  \begin{Def}
    A \emph{binary operation} on a set $S$ is a map $\mu:S \times S \to S$.
    Similarly, a \emph{unary operation} is a map $\mu:S \to S$.
  \end{Def}

  \begin{Def}
    A \emph{semigroup} is a set $S$ with a binary operation $*$ satisfying
    the associative law:
    \[
      a * (b * c) = (a * b) * c \text{\qquad for all } a, b,c \in S
    \]

    By the definition of the binary operation, the semigroup is closed with
    respect to the $*$ operator.

  \end{Def}

  \begin{Def} A \emph{group} is a non-empty set $G$ with a binary
    operation $*$, a unary operation $x \to \bar{x}$,
    and an element $e$, such that:
    \begin{enumerate}
      \item $G$ is a semigroup with respect to $*$,
      \item $a * e = e * a = a$ for all $a \in G$, and
      \item $a * \bar{a} = \bar{a} * a = e$ for all $a \in G$.
    \end{enumerate}
    $e$ is the identity element of $G$, and $\bar{a}$ is the inverse of $a$.
    If the $*$ operation is commutative, the group is known as an
    \emph{Abelian} group.

  \end{Def}

  \begin{Lemma}
    Both the identity element of a group, and the inverse of a particular
    element are uniquely determined.
  \end{Lemma}

  \begin{proof}
    Suppose $e$ is an identity element of a group $G$. Let $a$ be an identity.
    Then we have $a = ae$, since $e$ is an identity, and because $a$ is an
    identity, $ae = e$, and so $a = e$, uniquely determined the identity.

    For the second part, let $a^{-1}$ be the inverse of the element $a \in G$,
    and suppose $b$ is also the inverse of a. Then we have $a * b = e = a *
    a^{-1}$, and $ a^{-1} * a * b = a^{-1} * a * a^{-1} $, so $ (a^{-1} * a) *
    b = (a^{-1} * a) * a^{-1} $ and hence $b = a^{-1}$ and the inverse is
    uniquely determined.
  \end{proof}

  \begin{Def}
    A \emph{ring} is a set $R$ equipped with two binary operations connected by
    distributive laws. $R$ is an Abelian group with respect to one binary
    operation (by convention known as addition and denoted by $+$), and is a
    semigroup with respect to the other binary operation (called multiplication
    and denoted by juxtaposition).

    The full set of ring axioms are as follows for a ring $R$:
    \begin{enumerate}
        \item
          $ a + b = b + a $ for all $ a, b \in R$,
        \item
          $ a + (b + c) = (a + b) + c $ for all $ a, b, c \in R$,
        \item
          There is an element $0 \in R$, such that
          $a + 0 = 0 + a = a$ for all $a \in R$,
        \item
          For each $a \in R$, there exists an element $-a$ such that
          $a + -a = -a + a = 0$,

        \item
          $ a(bc) = (ab)c $ for all $ a, b, c \in R$,

        \item
          $ a(b+c) = ab + ac $ for all $ a, b, c \in R$,
        \item
          $ (a+b)c = ac + bc $ for all $ a, b, c \in R$,

    \end{enumerate}
  \end{Def}

  \begin{Lemma}
    Some consequences of the ring axioms:
    \begin{align}
      r0 &= 0r = 0,\\
      (-r)s &= r(-s) = -(rs),\\
      (-r)(-s) &= rs.
    \end{align}
    for all $r, s \in R$.
  \end{Lemma}

  \begin{proof}
    Since $0$ is the additive identity, we have $0 + 0 = 0$,
    hence $r(0+0) = r0$, and by the left associative law
    $ r0 + r0 = r0 $, and $r0 + r0 = r0 + 0$. Then by the
    cancellation law, $r0 = 0$. Similarly, $0r = 0$.

    Then, given $-r$ is the additive inverse of $r$, $r + (-r) = 0$ and so
    $(r + (-r))s = 0s$, and by the right distributive law and the previous
    identity, $rs + (-r)s = 0$. Now $rs + -(rs) = 0$, so
    $rs + (-r)s = rs + -(rs)$ and by the cancellation law, $(-r)s = -(rs)$.
    Similarly, $r(-s) = -(rs)$.

    By applying (2) repeatedly, we have $(-r)(-s) = -(r(-s)) = --(rs)$.
    Now, for any $t \in R$, $-t$ is the unique solution to $t + x = 0$.
    So $(-t) + t = 0$ tells us $--t = t$, and so $(-r)(-s) = rs$.
  \end{proof}

  \subsection{Special Classes of Rings}

  \begin{Def}
    A \emph{Commutative Ring} is a ring in which multiplication is commutative,
    i.e. $ab = ba$ for all $a, b \in R$.
  \end{Def}

  \begin{Def}
    A \emph{Ring with Multiplicative Identity} or \emph{Ring with 1} is
    a ring containing the element $1$, where $1a = a1 = a$ for all
    $a \in R$.
  \end{Def}

  Note that the ring $\lbrace 0 \rbrace$ is a ring with 1, and `1' in this case is $0$.

  \begin{Lemma}
    The multiplicative identity in a ring with 1 is uniquely determined
  \end{Lemma}
  \begin{proof}
    Let $e$ be the multiplicative identity. Suppose there is an $x \in R$ such
    that $rx = xr = r$ for all $r \in R$, then $x = ex = e$
  \end{proof}

  \begin{Def}
    A \emph{zero-divisor} of a ring $R$ is an element $r \neq 0 \in R$ such
    that $rs = 0$ for some $s \neq 0 \in R$.
  \end{Def}

  \begin{Def}
    An \emph{Integral Domain} is a commutative ring with $1 \neq 0$ and no zero-divisors.
  \end{Def}

  \begin{Lemma}
    Let $a$ be a non-zero element in an integral domain, and $x, y \in R$. Then
    $$ax = ay \implies x = y$$
  \end{Lemma}

  \begin{proof}
    If $ax = ay$ for $a \neq 0 \in R$ where $R$ is an integral domain, then
    $a(x-y) = 0$. Since $R$ has no zero-divisors and $a \neq 0$, $x-y = 0$,
    and so $x = y$.
  \end{proof}

  \begin{Def}
    A \emph{field} is a commutative ring in which the set of non-zero elements
    forms a group under multiplication.
  \end{Def}

  \begin{Lemma}
    A field is an integral domain.
  \end{Lemma}

  \begin{proof}
    We need to show a field $F$ is a ring with $1$, and that there are no zero-divisors.

    Now since the non-zero elements of $F$ form a group under multiplication,
    there exists an element $1 \in F$ such that $1f = f1 = f$ for all $f \neq 0 \in F$.
    Since $1 \cdot 0 = 0$ as shown above, $1$ is indeed the multiplicative identity.

    Now suppose $ab = 0$ for some $a, b \in F, a \neq 0$. Since non-zero elements of $F$
    form a group under multiplication, there exists an element $a^{-1}$ such that
    $a^{-1}a = 1$. So $a^{-1}ab = a^{-1}0$, hence $b = 0$. Thus $F$ has no zero-divisors.
  \end{proof}

  \subsection{Examples of rings}

  \subsubsection{$n\mathbb{Z}$}
  If $n \in \mathbb{Z}$, the set $n\mathbb{Z} = \lbrace a \in \mathbb{Z} : n | a \rbrace$
  is a commutative ring.

  \subsubsection{$\mathbb{Z}_{n}$}
  Given $n \in \mathbb{Z}$, define the equivalence relation $\sim$ by
  \[
    a \sim b \iff a - b = kn
  \]
  for some $k \in \mathbb{Z}$.

  Denote the equivalence class containing $a$ as $[a]$.
  Then $[0], [1], ..., [n-1]$ is the complete set of equivalence classes with respect
  to $\sim$. These are called the \emph{congruence classes modulo $n$}, and the set of
  them is denoted $\mathbb{Z}_{n}$.

  When we define the operators as follows:
  \begin{align*}
    [a] + [b] &= [a + b]\\
    [a][b] &= [ab]
  \end{align*}
  we get a commutative ring with $1$. If $n$ is composite, we have zero-divisors, so
  the ring is not an integral domain, but if $n$ is prime, $\mathbb{Z}_{n}$ is a field.

  \subsubsection{Abelian group with null multiplication}
  Let $G$ be an Abelian group, and define multiplication as $ab = 0$ for all $a, b \in G$,
  then it forms a commutative ring.

  \subsubsection{$\mathbb{C}$}
  The complex numbers form a field under the normal operations. The subrings
  $\mathbb{R}$ and $\mathbb{Q}$ also form fields, and the subring $\mathbb{Z}$
  is an integral domain.

  \subsubsection{Gaussian integers}
  ${a + bi : a, b \in \mathbb{Z}}$ forms an integral domain under the normal operations.

  \subsubsection{Power set}
  Given a set $S$, its power set is defined as the set of all subsets of $S$,
  including $S$ and the empty set. Define the operations as
  \begin{align*}
    A + B &= (A \cup B) \setminus (A \cap B)\\
    AB &= A \cap B
  \end{align*}
  and we get a commutative ring.

  \subsubsection{The set of $n \times n$ matrices}
  Let $\mathbf{M}_{n}(\mathbf{k})$ denote the set of $n \times n$ matrices over
  a field $\mathbf{k}$. The operations are defined the normal way as $(A +
  B)_{ij} = A_{ij} + B_{ij}$, and $(AB)_{ij} = \sum_{k=1}^{n} A_{ik} B_{kj}$,
  where $A_{ij}$ is the $i$th row and $j$th column of the matrix $A$. Then
  $\mathbf{M}_{n}(\mathbf{k})$ forms a ring with $1$. Note, it is not
  commutative when $n > 1$.

  \subsubsection{The set of all maps $\phi : S \to \mathbb{R}$}


  %\subsubsection


\section{Subrings, Homomorphisms and Ideals}

\subsection{Subrings}

\begin{Def}
  A \emph{subring} $S$ of a ring $R$ is a subset of $R$ which is a ring under
  the operations it inherits from $R$.
\end{Def}

\begin{Lemma}
  Let $S$ be a subset of a ring $R$. Then $S$ is a subring of $R$ if and only
  if (i) $S$ is non-empty, and (ii) whenever $a, b \in S$ then $a - b \in S$
  and $ab \in S$.
\end{Lemma}

\begin{proof}
  Let $S$ be a subring of $R$, then $S$ is a group under $+$ and so is
  non-empty. If $a, b \in S$, then $-b \in S$, from the group additive inverse
  axiom, and $a + -b \in S$, hence $a - b \in S$ as $+$ is a binary operation
  $+: S \times S \to S$.  Multiplication is also a binary operation on $S$, and
  therefore if $a, b \in S$ then $ab \in S$ also.

  For the sufficiency, we have $S$ is a non-empty subset of $R$, and so
  contains an element $a$. It also contains $a - a = 0$.  So if $b \in S$, then
  $0 - b = -b \in S$, and $a - (-b) = a + b \in S$.

  The commutative and associative laws of addition are consequences of those in
  $R$, since when we add elements of $S$ we do so by thinking of them as
  elements of $R$. Hence $S$ is an Abelian group under $+$ with identity
  element $0$.

  The associative law of multiplication and the two distributive laws follow in
  the same way from those in $R$. Hence $S$ is a ring.
\end{proof}

\begin{Def}
  The \emph{additive group} of a ring $R$, denoted by $R^+$, is $R$ with the
  binary operation of multiplication omitted. Subgroups of $R^+$ are often
  called \emph{additive subgroups} of $R$.
\end{Def}

\begin{Def}
  If $A$ is any Abelian group written additively, $a \in A$ and $n \in
  \mathbb{Z}$, then
  \[
    na = 
    \begin{cases}
      a + \cdots + a \text{ (with $n$ terms $a$),} &\text{if $n > 0$;}\\
      0, &\text{if $n = 0$;}\\
      (-n)(-a) = -(a + \cdots + a) \text{ (with $|n|$ terms),} &\text{if $n < 0$.}
    \end{cases}
  \]

  For $a, b \in A$ and $n, m \in \mathbb{Z}$, we have
  \begin{align*}
    n(a+b) &= na + nb,\\
    (n+m)a &= na + ma,\\
    (nm)a &= n(ma),\\
    1a &= a.
  \end{align*}
\end{Def}

\begin{Def}
  Let $S, T$ be arbitrary non-empty subsets of a ring $R$. We define
  \begin{align*}
    S + T &= \{ s+t: s \in S, t \in T \}\\
    ST &= \Bigl\{\sum^{n}_{i = 1} s_i t_i : s_i \in S, t_i \in T, n = 1, 2, \dots \Bigr\} .
  \end{align*}

\end{Def}

\begin{Lemma}
  Let $R$ be a ring, and $S, T, U$ be non-empty subsets of R. Then:
  \begin{enumerate}[label=(\roman*)]
    \item $(S + T) + U = S + (T + U)$ and $(ST)U = S(TU)$.
    \item If $S$ and $T$ are additive subgroups of $R$, then so are $S+T$ and
      $ST$.
    \item If $S$ and $T$ are subrings of R and R is commutative, then $ST$ is a
      subring of $R$.
  \end{enumerate}
\end{Lemma}

\begin{proof}
  (i) Additive associativity is given from the associativity of ring operation.
  Now, since $ST$ consists of all finite sums of elements of the form $st$ with
  $s \in S$ and $t \in T$, it is closed under addition. Therefore so are $(ST)U$
  and $S(TU)$. Now, an arbitrary element $z$ of $(ST)U$ is a finte sum of elements
  of the form $xu$ with $x \in ST$ and $u \in U$. $x$ is a finite sum of elements
  of the form $st$ with $s \in S$ and $t \in T$, and $z$ is a sum of elements
  $(st)u$. Since $(st)u = s(tu)$, these elements all belong to $S(TU)$. Since
  $S(TU)$ is closed under addition, $z \in S(TU)$. Therefore $(ST)U \subseteq
  S(TU)$ and the reverse inclusion can be established similarly.

  (ii) Let $x, x' \in S + T$. Then $x = s + t, x' = s' + t'$ for some $s, s' \in S$
  and $t, t' \in T$. Therefore $x - x' = (s - s') + (t - t') \in S + T$ as $S$
  and $T$ are additive groups. Further $0$ belongs to both $S$ and $T$, and so 
  $0 = 0 + 0 \in S + T$. Hence $S + T$ is an additive subgroup of $R$.

  Now ST is closed under addition (see above). If $y = \sum s_i t_i \in ST$, then
  $-y = \sum (-s_i)t_i \in ST$ since $-s_i \in S$. Since $ST$ clearly contains $0$,
  it is therefore an additive subgroup of $R$.

  (iii) $ST$ is a subgroup of $R$. However $(\sum_i s_i t_i)(\sum_j s'_j t'_j) =
  \sum_{ij}(s_i s'_j)(t_j t'_j)$, as $R$ is commutative; and this element belongs
  to $ST$.
\end{proof}

\subsection{Homomorphisms}

\begin{Def}
  A \emph{homomorphism} of a ring $R$ into a ring $S$ is a map
  $\phi : R \to S$ such that
  \begin{enumerate}[label=(\roman*)]
    \item $\phi(x+y) = \phi(x) + \phi(y)$, and
    \item $\phi(xy) = \phi(x) \phi(y)$
  \end{enumerate}
  for all $x, y \in R$.
\end{Def}

By the first rule, $\phi$ is a group homomorphism $R^+ \to S^+$, and so
$\phi(0_R) = 0_S$ and $\phi(-r) = -\phi(r)$ for all $r \in R$.

\begin{Def}
  Various names of special ring homomorphisms:

  An \emph{epimorphism} $R \to S$ is a surjective homomorphism.

  A \emph{monomorphism} $R \to S$ is an injective homomorphism.

  An \emph{isomorphism} $R \to S$ is a bijective homomorphism.

  An \emph{endomorphism} $R \to R$ is a homomorphism of $R$ into itself.

  An \emph{automorphism} $R \to R$ is an isomorphism of $R$ into itself.
\end{Def}

\begin{Lemma}
The composition of two homomorphisms is a homomorphism; the composition
of two epimorphisms is an epimorphism; and the composition of two
monomorphisms is a monomorphism.
\end{Lemma}

\begin{proof}
  Let $\phi : R \to S$ and $\psi : S \to T$ be homomorphisms. Then $\phi(r_1 +
  r_2) = \phi(r_1) + \phi(r_2)$ and $\psi(s_1 + s_2) = \psi(s_1) + \psi(s_2)$
  for all $r_1, r_2 \in R, s_1, s_2 \in S$.  Now $\phi(r_1) = s_1$ and
  $\phi(r_2) = s_2$ for some $r_1, r_2 \in R$ and $s_1, s_2 \in S$. So
  $\psi(\phi(r_1 + r_2)) = \psi(\phi(r_1) + \phi(r_2)) = \psi(s_1 + s_2) =
  \psi(s_1) + \psi(s_2) = \psi(\phi(r_1)) + \psi(\phi(r_2))$.  Similarly,
  $\psi(\phi(r_1 r_2)) = \psi(\phi(r_1)) \psi(\phi(r_2))$.

  Now if $\phi$ and $\psi$ are epimorphisms, the image of $\phi$ is $S$, and
  the image of $\psi$ is $T$. And $\psi \phi : R \to T$ has image $T$ and is
  therefore also an epimorphism.

  If $\phi$ and $\psi$ are monomorphisms, then $t = \psi(s)$ for some unique $s
  \in S$, and $s = \phi(r)$ for some unique $r \in R$. Therefore $t = \psi (
  \phi (r) )$ for some unique $r \in R$ and the composition is a monomorphism.
\end{proof}

\begin{Cor}
  The composition of two isomorphisms is an isomorphism, and similarly for
  automorphisms and endomorphisms.
\end{Cor}

\begin{proof}
  This follows immediately from the previous lemma.
\end{proof}

\begin{Lemma}
  If $\phi : R \to S$ is an isomorphism, its inverse $\phi^{-1} : S \to R$
  is also an isomorphism.
\end{Lemma}

\begin{proof}
  $\phi^{-1}$ exists, since $\phi$ is bijective. If $s, s' \in S$, then $s =
  \phi(r), s' = \phi(r')$ for some $r, r' \in R$. So $\phi^{-1}(ss') =
  \phi^{-1}(\phi(r)\phi(r')) = \phi^{-1}(\phi(rr')) = rr' =
  \phi^{-1}(s)\phi^{-1}(s')$. Similarly $\phi^{-1}(ss') =
  \phi^{-1}(s)\phi^{-1}(s')$.
\end{proof}

\begin{Def}
  If there exists an isomorphism from $R$ to $S$, we write $R \cong S$
  and say $R$ is \emph{isomorphic} to $S$.
\end{Def}

The symbol $\cong$ is an equivalence relation, i.e. (i) $R \cong R$, (ii) $R
\cong S \implies S \cong R$, (iii) $R \cong S$ and $S \cong T \implies R \cong
T$.  This can be deduced from the isomorphism discussion above.



\end{document}
