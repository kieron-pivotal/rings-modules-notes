\section{Subrings, Homomorphisms and Ideals}

\subsection{Subrings}

\begin{Def}
  A \emph{subring} $S$ of a ring $R$ is a subset of $R$ which is a ring under
  the operations it inherits from $R$.
\end{Def}

\begin{Lemma}
  Let $S$ be a subset of a ring $R$. Then $S$ is a subring of $R$ if and only
  if (i) $S$ is non-empty, and (ii) whenever $a, b \in S$ then $a - b \in S$
  and $ab \in S$.
\end{Lemma}

\begin{proof}
  Let $S$ be a subring of $R$, then $S$ is a group under $+$ and so is
  non-empty. If $a, b \in S$, then $-b \in S$, from the group additive inverse
  axiom, and $a + -b \in S$, hence $a - b \in S$ as $+$ is a binary operation
  $+: S \times S \to S$.  Multiplication is also a binary operation on $S$, and
  therefore if $a, b \in S$ then $ab \in S$ also.

  For the sufficiency, we have $S$ is a non-empty subset of $R$, and so
  contains an element $a$. It also contains $a - a = 0$.  So if $b \in S$, then
  $0 - b = -b \in S$, and $a - (-b) = a + b \in S$.

  The commutative and associative laws of addition are consequences of those in
  $R$, since when we add elements of $S$ we do so by thinking of them as
  elements of $R$. Hence $S$ is an Abelian group under $+$ with identity
  element $0$.

  The associative law of multiplication and the two distributive laws follow in
  the same way from those in $R$. Hence $S$ is a ring.
\end{proof}

\begin{Def}
  The \emph{additive group} of a ring $R$, denoted by $R^+$, is $R$ with the
  binary operation of multiplication omitted. Subgroups of $R^+$ are often
  called \emph{additive subgroups} of $R$.
\end{Def}

\begin{Def}
  If $A$ is any Abelian group written additively, $a \in A$ and $n \in
  \mathbb{Z}$, then
  \[
    na = 
    \begin{cases}
      a + \cdots + a \text{ (with $n$ terms $a$),} &\text{if $n > 0$;}\\
      0, &\text{if $n = 0$;}\\
      (-n)(-a) = -(a + \cdots + a) \text{ (with $|n|$ terms),} &\text{if $n < 0$.}
    \end{cases}
  \]

  For $a, b \in A$ and $n, m \in \mathbb{Z}$, we have
  \begin{align*}
    n(a+b) &= na + nb,\\
    (n+m)a &= na + ma,\\
    (nm)a &= n(ma),\\
    1a &= a.
  \end{align*}
\end{Def}

\begin{Def}
  Let $S, T$ be arbitrary non-empty subsets of a ring $R$. We define
  \begin{align*}
    S + T &= \{ s+t: s \in S, t \in T \}\\
    ST &= \Bigl\{\sum^{n}_{i = 1} s_i t_i : s_i \in S, t_i \in T, n = 1, 2, \dots \Bigr\} .
  \end{align*}

\end{Def}

\begin{Lemma}
  Let $R$ be a ring, and $S, T, U$ be non-empty subsets of R. Then:
  \begin{enumerate}[label=(\roman*)]
    \item $(S + T) + U = S + (T + U)$ and $(ST)U = S(TU)$.
    \item If $S$ and $T$ are additive subgroups of $R$, then so are $S+T$ and
      $ST$.
    \item If $S$ and $T$ are subrings of R and R is commutative, then $ST$ is a
      subring of $R$.
  \end{enumerate}
\end{Lemma}

\begin{proof}
  (i) Additive associativity is given from the associativity of ring operation.
  Now, since $ST$ consists of all finite sums of elements of the form $st$ with
  $s \in S$ and $t \in T$, it is closed under addition. Therefore so are $(ST)U$
  and $S(TU)$. Now, an arbitrary element $z$ of $(ST)U$ is a finte sum of elements
  of the form $xu$ with $x \in ST$ and $u \in U$. $x$ is a finite sum of elements
  of the form $st$ with $s \in S$ and $t \in T$, and $z$ is a sum of elements
  $(st)u$. Since $(st)u = s(tu)$, these elements all belong to $S(TU)$. Since
  $S(TU)$ is closed under addition, $z \in S(TU)$. Therefore $(ST)U \subseteq
  S(TU)$ and the reverse inclusion can be established similarly.

  (ii) Let $x, x' \in S + T$. Then $x = s + t$, $x' = s' + t'$ for some $s, s' \in S$
  and $t, t' \in T$. Therefore $x - x' = (s - s') + (t - t') \in S + T$ as $S$
  and $T$ are additive groups. Further $0$ belongs to both $S$ and $T$, and so 
  $0 = 0 + 0 \in S + T$. Hence $S + T$ is an additive subgroup of $R$.

  Now ST is closed under addition (see above). If $y = \sum s_i t_i \in ST$, then
  $-y = \sum (-s_i)t_i \in ST$ since $-s_i \in S$. Since $ST$ clearly contains $0$,
  it is therefore an additive subgroup of $R$.

  (iii) $ST$ is a subgroup of $R$. However $(\sum_i s_i t_i)(\sum_j s'_j t'_j) =
  \sum_{ij}(s_i s'_j)(t_j t'_j)$, as $R$ is commutative; and this element belongs
  to $ST$.
\end{proof}

\subsection{Homomorphisms}

\begin{Def}
  A \emph{homomorphism} of a ring $R$ into a ring $S$ is a map
  $\phi : R \to S$ such that
  \begin{enumerate}[label=(\roman*)]
    \item $\phi(x+y) = \phi(x) + \phi(y)$, and
    \item $\phi(xy) = \phi(x) \phi(y)$
  \end{enumerate}
  for all $x, y \in R$.
\end{Def}

By the first rule, $\phi$ is a group homomorphism $R^+ \to S^+$, and so
$\phi(0_R) = 0_S$ and $\phi(-r) = -\phi(r)$ for all $r \in R$.

\begin{Def}
  Various names of special ring homomorphisms:

  An \emph{epimorphism} $R \to S$ is a surjective homomorphism.

  A \emph{monomorphism} $R \to S$ is an injective homomorphism.

  An \emph{isomorphism} $R \to S$ is a bijective homomorphism.

  An \emph{endomorphism} $R \to R$ is a homomorphism of $R$ into itself.

  An \emph{automorphism} $R \to R$ is an isomorphism of $R$ into itself.
\end{Def}

\begin{Lemma}
The composition of two homomorphisms is a homomorphism; the composition
of two epimorphisms is an epimorphism; and the composition of two
monomorphisms is a monomorphism.
\end{Lemma}

\begin{proof}
  Let $\phi : R \to S$ and $\psi : S \to T$ be homomorphisms. Then $\phi(r_1 +
  r_2) = \phi(r_1) + \phi(r_2)$ and $\psi(s_1 + s_2) = \psi(s_1) + \psi(s_2)$
  for all $r_1, r_2 \in R$, $s_1, s_2 \in S$.  Now $\phi(r_1) = s_1$ and
  $\phi(r_2) = s_2$ for some $r_1, r_2 \in R$ and $s_1, s_2 \in S$. So
  $\psi(\phi(r_1 + r_2)) = \psi(\phi(r_1) + \phi(r_2)) = \psi(s_1 + s_2) =
  \psi(s_1) + \psi(s_2) = \psi(\phi(r_1)) + \psi(\phi(r_2))$.  Similarly,
  $\psi(\phi(r_1 r_2)) = \psi(\phi(r_1)) \psi(\phi(r_2))$.

  Now if $\phi$ and $\psi$ are epimorphisms, the image of $\phi$ is $S$, and
  the image of $\psi$ is $T$. And $\psi \phi : R \to T$ has image $T$ and is
  therefore also an epimorphism.

  If $\phi$ and $\psi$ are monomorphisms, then $t = \psi(s)$ for some unique $s
  \in S$, and $s = \phi(r)$ for some unique $r \in R$. Therefore $t = \psi (
  \phi (r) )$ for some unique $r \in R$ and the composition is a monomorphism.
\end{proof}

\begin{Cor}
  The composition of two isomorphisms is an isomorphism, and similarly for
  automorphisms and endomorphisms.
\end{Cor}

\begin{proof}
  This follows immediately from the previous lemma.
\end{proof}

\begin{Lemma}
  If $\phi : R \to S$ is an isomorphism, its inverse $\phi^{-1} : S \to R$
  is also an isomorphism.
\end{Lemma}

\begin{proof}
  $\phi^{-1}$ exists, since $\phi$ is bijective.  Now if $s, s' \in S$, then $s =
  \phi(r)$, $s' = \phi(r')$ for some $r, r' \in R$.  So $\phi^{-1}(ss') =
  \phi^{-1}(\phi(r)\phi(r')) = \phi^{-1}(\phi(rr')) = rr' =
  \phi^{-1}(s)\phi^{-1}(s')$. Similarly $\phi^{-1}(ss') =
  \phi^{-1}(s)\phi^{-1}(s')$.
\end{proof}

\begin{Def}
  If there exists an isomorphism from $R$ to $S$, we write $R \cong S$
  and say $R$ is \emph{isomorphic} to $S$.
\end{Def}

The symbol $\cong$ is an equivalence relation, i.e. (i) $R \cong R$, (ii) $R
\cong S \implies S \cong R$, (iii) $R \cong S$ and $S \cong T \implies R \cong
T$.  This can be deduced from the isomorphism discussion above.

