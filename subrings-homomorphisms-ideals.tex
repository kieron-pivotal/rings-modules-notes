\section{Subrings, Homomorphisms and Ideals}

\subsection{Subrings}

\begin{Def}
A \emph{subring} $S$ of a ring $R$ is a subset of $R$
which is a ring under the operations it inherits from $R$.
\end{Def}

\begin{Lemma}
Let $S$ be a subset of a ring $R$. Then $S$ is a subring
of $R$ if and only if (i) $S$ is non-empty, and (ii) whenever
$a, b \in S$ then $a - b \in S$ and $ab \in S$.
\end{Lemma}

\begin{proof}
Let $S$ be a subring of $R$, then $S$ is a group under $+$
and so is non-empty. If $a, b \in S$, then $-b \in S$, from
the group additive inverse axiom, and $a + -b \in S$, hence
$a - b \in S$ as $+$ is a binary operation $+: S \times S \to S$.
Multiplication is also a binary operation on $S$, and therefore
if $a, b \in S$ then $ab \in S$ also.

For the sufficiency, we have $S$ is a non-empty subset of $R$,
and so contains an element $a$. It also contains $a - a = 0$.
So if $b \in S$, then $0 - b = -b \in S$, and $a - (-b) = a + b
\in S$.

The commutative and associative laws of addition are consequences
of those in $R$, since when we add elements of $S$ we do so by
thinking of them as elements of $R$. Hence $S$ is an Abelian
group under $+$ with identity element $0$.

The associative law of multiplication and the two distributive
laws follow in the same way from those in $R$. Hence $S$ is a ring.
\end{proof}

